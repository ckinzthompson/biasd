% Generated by Sphinx.
\def\sphinxdocclass{report}
\documentclass[letterpaper,10pt,english]{sphinxmanual}

\usepackage[utf8]{inputenc}
\ifdefined\DeclareUnicodeCharacter
  \DeclareUnicodeCharacter{00A0}{\nobreakspace}
\else\fi
\usepackage{cmap}
\usepackage[T1]{fontenc}
\usepackage{amsmath,amssymb}
\usepackage{babel}
\usepackage{times}
\usepackage[Bjarne]{fncychap}
\usepackage{longtable}
\usepackage{sphinx}
\usepackage{multirow}
\usepackage{eqparbox}


\addto\captionsenglish{\renewcommand{\figurename}{Fig. }}
\addto\captionsenglish{\renewcommand{\tablename}{Table }}
\SetupFloatingEnvironment{literal-block}{name=Listing }

\addto\extrasenglish{\def\pageautorefname{page}}

\setcounter{tocdepth}{1}


\title{BIASD Documentation}
\date{Jun 15, 2016}
\release{0.1}
\author{Colin Kinz-Thompson}
\newcommand{\sphinxlogo}{}
\renewcommand{\releasename}{Release}
\makeindex

\makeatletter
\def\PYG@reset{\let\PYG@it=\relax \let\PYG@bf=\relax%
    \let\PYG@ul=\relax \let\PYG@tc=\relax%
    \let\PYG@bc=\relax \let\PYG@ff=\relax}
\def\PYG@tok#1{\csname PYG@tok@#1\endcsname}
\def\PYG@toks#1+{\ifx\relax#1\empty\else%
    \PYG@tok{#1}\expandafter\PYG@toks\fi}
\def\PYG@do#1{\PYG@bc{\PYG@tc{\PYG@ul{%
    \PYG@it{\PYG@bf{\PYG@ff{#1}}}}}}}
\def\PYG#1#2{\PYG@reset\PYG@toks#1+\relax+\PYG@do{#2}}

\expandafter\def\csname PYG@tok@gd\endcsname{\def\PYG@tc##1{\textcolor[rgb]{0.63,0.00,0.00}{##1}}}
\expandafter\def\csname PYG@tok@gu\endcsname{\let\PYG@bf=\textbf\def\PYG@tc##1{\textcolor[rgb]{0.50,0.00,0.50}{##1}}}
\expandafter\def\csname PYG@tok@gt\endcsname{\def\PYG@tc##1{\textcolor[rgb]{0.00,0.27,0.87}{##1}}}
\expandafter\def\csname PYG@tok@gs\endcsname{\let\PYG@bf=\textbf}
\expandafter\def\csname PYG@tok@gr\endcsname{\def\PYG@tc##1{\textcolor[rgb]{1.00,0.00,0.00}{##1}}}
\expandafter\def\csname PYG@tok@cm\endcsname{\let\PYG@it=\textit\def\PYG@tc##1{\textcolor[rgb]{0.25,0.50,0.56}{##1}}}
\expandafter\def\csname PYG@tok@vg\endcsname{\def\PYG@tc##1{\textcolor[rgb]{0.73,0.38,0.84}{##1}}}
\expandafter\def\csname PYG@tok@m\endcsname{\def\PYG@tc##1{\textcolor[rgb]{0.13,0.50,0.31}{##1}}}
\expandafter\def\csname PYG@tok@mh\endcsname{\def\PYG@tc##1{\textcolor[rgb]{0.13,0.50,0.31}{##1}}}
\expandafter\def\csname PYG@tok@cs\endcsname{\def\PYG@tc##1{\textcolor[rgb]{0.25,0.50,0.56}{##1}}\def\PYG@bc##1{\setlength{\fboxsep}{0pt}\colorbox[rgb]{1.00,0.94,0.94}{\strut ##1}}}
\expandafter\def\csname PYG@tok@ge\endcsname{\let\PYG@it=\textit}
\expandafter\def\csname PYG@tok@vc\endcsname{\def\PYG@tc##1{\textcolor[rgb]{0.73,0.38,0.84}{##1}}}
\expandafter\def\csname PYG@tok@il\endcsname{\def\PYG@tc##1{\textcolor[rgb]{0.13,0.50,0.31}{##1}}}
\expandafter\def\csname PYG@tok@go\endcsname{\def\PYG@tc##1{\textcolor[rgb]{0.20,0.20,0.20}{##1}}}
\expandafter\def\csname PYG@tok@cp\endcsname{\def\PYG@tc##1{\textcolor[rgb]{0.00,0.44,0.13}{##1}}}
\expandafter\def\csname PYG@tok@gi\endcsname{\def\PYG@tc##1{\textcolor[rgb]{0.00,0.63,0.00}{##1}}}
\expandafter\def\csname PYG@tok@gh\endcsname{\let\PYG@bf=\textbf\def\PYG@tc##1{\textcolor[rgb]{0.00,0.00,0.50}{##1}}}
\expandafter\def\csname PYG@tok@ni\endcsname{\let\PYG@bf=\textbf\def\PYG@tc##1{\textcolor[rgb]{0.84,0.33,0.22}{##1}}}
\expandafter\def\csname PYG@tok@nl\endcsname{\let\PYG@bf=\textbf\def\PYG@tc##1{\textcolor[rgb]{0.00,0.13,0.44}{##1}}}
\expandafter\def\csname PYG@tok@nn\endcsname{\let\PYG@bf=\textbf\def\PYG@tc##1{\textcolor[rgb]{0.05,0.52,0.71}{##1}}}
\expandafter\def\csname PYG@tok@no\endcsname{\def\PYG@tc##1{\textcolor[rgb]{0.38,0.68,0.84}{##1}}}
\expandafter\def\csname PYG@tok@na\endcsname{\def\PYG@tc##1{\textcolor[rgb]{0.25,0.44,0.63}{##1}}}
\expandafter\def\csname PYG@tok@nb\endcsname{\def\PYG@tc##1{\textcolor[rgb]{0.00,0.44,0.13}{##1}}}
\expandafter\def\csname PYG@tok@nc\endcsname{\let\PYG@bf=\textbf\def\PYG@tc##1{\textcolor[rgb]{0.05,0.52,0.71}{##1}}}
\expandafter\def\csname PYG@tok@nd\endcsname{\let\PYG@bf=\textbf\def\PYG@tc##1{\textcolor[rgb]{0.33,0.33,0.33}{##1}}}
\expandafter\def\csname PYG@tok@ne\endcsname{\def\PYG@tc##1{\textcolor[rgb]{0.00,0.44,0.13}{##1}}}
\expandafter\def\csname PYG@tok@nf\endcsname{\def\PYG@tc##1{\textcolor[rgb]{0.02,0.16,0.49}{##1}}}
\expandafter\def\csname PYG@tok@si\endcsname{\let\PYG@it=\textit\def\PYG@tc##1{\textcolor[rgb]{0.44,0.63,0.82}{##1}}}
\expandafter\def\csname PYG@tok@s2\endcsname{\def\PYG@tc##1{\textcolor[rgb]{0.25,0.44,0.63}{##1}}}
\expandafter\def\csname PYG@tok@vi\endcsname{\def\PYG@tc##1{\textcolor[rgb]{0.73,0.38,0.84}{##1}}}
\expandafter\def\csname PYG@tok@nt\endcsname{\let\PYG@bf=\textbf\def\PYG@tc##1{\textcolor[rgb]{0.02,0.16,0.45}{##1}}}
\expandafter\def\csname PYG@tok@nv\endcsname{\def\PYG@tc##1{\textcolor[rgb]{0.73,0.38,0.84}{##1}}}
\expandafter\def\csname PYG@tok@s1\endcsname{\def\PYG@tc##1{\textcolor[rgb]{0.25,0.44,0.63}{##1}}}
\expandafter\def\csname PYG@tok@gp\endcsname{\let\PYG@bf=\textbf\def\PYG@tc##1{\textcolor[rgb]{0.78,0.36,0.04}{##1}}}
\expandafter\def\csname PYG@tok@sh\endcsname{\def\PYG@tc##1{\textcolor[rgb]{0.25,0.44,0.63}{##1}}}
\expandafter\def\csname PYG@tok@ow\endcsname{\let\PYG@bf=\textbf\def\PYG@tc##1{\textcolor[rgb]{0.00,0.44,0.13}{##1}}}
\expandafter\def\csname PYG@tok@sx\endcsname{\def\PYG@tc##1{\textcolor[rgb]{0.78,0.36,0.04}{##1}}}
\expandafter\def\csname PYG@tok@bp\endcsname{\def\PYG@tc##1{\textcolor[rgb]{0.00,0.44,0.13}{##1}}}
\expandafter\def\csname PYG@tok@c1\endcsname{\let\PYG@it=\textit\def\PYG@tc##1{\textcolor[rgb]{0.25,0.50,0.56}{##1}}}
\expandafter\def\csname PYG@tok@kc\endcsname{\let\PYG@bf=\textbf\def\PYG@tc##1{\textcolor[rgb]{0.00,0.44,0.13}{##1}}}
\expandafter\def\csname PYG@tok@c\endcsname{\let\PYG@it=\textit\def\PYG@tc##1{\textcolor[rgb]{0.25,0.50,0.56}{##1}}}
\expandafter\def\csname PYG@tok@mf\endcsname{\def\PYG@tc##1{\textcolor[rgb]{0.13,0.50,0.31}{##1}}}
\expandafter\def\csname PYG@tok@err\endcsname{\def\PYG@bc##1{\setlength{\fboxsep}{0pt}\fcolorbox[rgb]{1.00,0.00,0.00}{1,1,1}{\strut ##1}}}
\expandafter\def\csname PYG@tok@mb\endcsname{\def\PYG@tc##1{\textcolor[rgb]{0.13,0.50,0.31}{##1}}}
\expandafter\def\csname PYG@tok@ss\endcsname{\def\PYG@tc##1{\textcolor[rgb]{0.32,0.47,0.09}{##1}}}
\expandafter\def\csname PYG@tok@sr\endcsname{\def\PYG@tc##1{\textcolor[rgb]{0.14,0.33,0.53}{##1}}}
\expandafter\def\csname PYG@tok@mo\endcsname{\def\PYG@tc##1{\textcolor[rgb]{0.13,0.50,0.31}{##1}}}
\expandafter\def\csname PYG@tok@kd\endcsname{\let\PYG@bf=\textbf\def\PYG@tc##1{\textcolor[rgb]{0.00,0.44,0.13}{##1}}}
\expandafter\def\csname PYG@tok@mi\endcsname{\def\PYG@tc##1{\textcolor[rgb]{0.13,0.50,0.31}{##1}}}
\expandafter\def\csname PYG@tok@kn\endcsname{\let\PYG@bf=\textbf\def\PYG@tc##1{\textcolor[rgb]{0.00,0.44,0.13}{##1}}}
\expandafter\def\csname PYG@tok@o\endcsname{\def\PYG@tc##1{\textcolor[rgb]{0.40,0.40,0.40}{##1}}}
\expandafter\def\csname PYG@tok@kr\endcsname{\let\PYG@bf=\textbf\def\PYG@tc##1{\textcolor[rgb]{0.00,0.44,0.13}{##1}}}
\expandafter\def\csname PYG@tok@s\endcsname{\def\PYG@tc##1{\textcolor[rgb]{0.25,0.44,0.63}{##1}}}
\expandafter\def\csname PYG@tok@kp\endcsname{\def\PYG@tc##1{\textcolor[rgb]{0.00,0.44,0.13}{##1}}}
\expandafter\def\csname PYG@tok@w\endcsname{\def\PYG@tc##1{\textcolor[rgb]{0.73,0.73,0.73}{##1}}}
\expandafter\def\csname PYG@tok@kt\endcsname{\def\PYG@tc##1{\textcolor[rgb]{0.56,0.13,0.00}{##1}}}
\expandafter\def\csname PYG@tok@sc\endcsname{\def\PYG@tc##1{\textcolor[rgb]{0.25,0.44,0.63}{##1}}}
\expandafter\def\csname PYG@tok@sb\endcsname{\def\PYG@tc##1{\textcolor[rgb]{0.25,0.44,0.63}{##1}}}
\expandafter\def\csname PYG@tok@k\endcsname{\let\PYG@bf=\textbf\def\PYG@tc##1{\textcolor[rgb]{0.00,0.44,0.13}{##1}}}
\expandafter\def\csname PYG@tok@se\endcsname{\let\PYG@bf=\textbf\def\PYG@tc##1{\textcolor[rgb]{0.25,0.44,0.63}{##1}}}
\expandafter\def\csname PYG@tok@sd\endcsname{\let\PYG@it=\textit\def\PYG@tc##1{\textcolor[rgb]{0.25,0.44,0.63}{##1}}}

\def\PYGZbs{\char`\\}
\def\PYGZus{\char`\_}
\def\PYGZob{\char`\{}
\def\PYGZcb{\char`\}}
\def\PYGZca{\char`\^}
\def\PYGZam{\char`\&}
\def\PYGZlt{\char`\<}
\def\PYGZgt{\char`\>}
\def\PYGZsh{\char`\#}
\def\PYGZpc{\char`\%}
\def\PYGZdl{\char`\$}
\def\PYGZhy{\char`\-}
\def\PYGZsq{\char`\'}
\def\PYGZdq{\char`\"}
\def\PYGZti{\char`\~}
% for compatibility with earlier versions
\def\PYGZat{@}
\def\PYGZlb{[}
\def\PYGZrb{]}
\makeatother

\renewcommand\PYGZsq{\textquotesingle}

\begin{document}

\maketitle
\tableofcontents
\phantomsection\label{index::doc}



\chapter{BIASD}
\label{index:bayesian-inference-for-the-analysis-of-sub-temporal-resolution-data}\label{index:biasd}
BIASD allows you to analyze Markovian signal versus time series, such as those collected in single-molecule biophysics experiments, even when the kinetics of the underlying Markov chain are faster than the signal acquisition rate. The code here has been written in python for easy implementation, but unfortunately, the likelihood function is computationally expensive since it involves a numerical integral. Therefore, the likelihood function is also provided as C code and also in CUDA with python wrappers to use them with the rest of the code base.


\chapter{Contents:}
\label{index:contents}

\section{Getting Started}
\label{getstarted:getstarted}\label{getstarted:getting-started}\label{getstarted::doc}
Here're some quick examples to get you started using BIASD. In general, BIASD uses the SMD data format (DOI: 10.1186/s12859-014-0429-4) for data storage, though this is not required. It also uses \titleref{emcee} (arXiv:1202.3665) to perform the Markov chain Monte Carlo (MCMC), though the Laplace approximation is also provided, which does not use emcee.

You can install \code{emcee} with

\begin{Verbatim}[commandchars=\\\{\}]
pip install emcee
\end{Verbatim}

You might also want to get \code{corner} for plotting purposes. Use

\begin{Verbatim}[commandchars=\\\{\}]
pip install corner
\end{Verbatim}


\subsection{BIASD + MCMC}
\label{getstarted:biasd-mcmc}
BIASD uses \titleref{emcee}, which is a seriously awesome, affine invariant Markov chain Monte Carlo sample. Read about it \href{http://dan.iel.fm/emcee/current/}{here}.

\begin{Verbatim}[commandchars=\\\{\}]
\PYG{k+kn}{import} \PYG{n+nn}{numpy} \PYG{k+kn}{as} \PYG{n+nn}{np}
\PYG{k+kn}{import} \PYG{n+nn}{biasd} \PYG{k+kn}{as} \PYG{n+nn}{b}

\PYG{c}{\PYGZsh{} Load some SMD format data}
\PYG{n}{data} \PYG{o}{=} \PYG{n}{b}\PYG{o}{.}\PYG{n}{smd}\PYG{o}{.}\PYG{n}{load}\PYG{p}{(}\PYG{l+s}{\PYGZsq{}}\PYG{l+s}{data.smd}\PYG{l+s}{\PYGZsq{}}\PYG{p}{)}

\PYG{c}{\PYGZsh{} Setup prior distributions}
\PYG{n}{e1} \PYG{o}{=} \PYG{n}{b}\PYG{o}{.}\PYG{n}{distributions}\PYG{o}{.}\PYG{n}{beta}\PYG{p}{(}\PYG{l+m+mi}{1}\PYG{p}{,}\PYG{l+m+mf}{9.}\PYG{p}{)}
\PYG{n}{e2} \PYG{o}{=} \PYG{n}{b}\PYG{o}{.}\PYG{n}{distributions}\PYG{o}{.}\PYG{n}{beta}\PYG{p}{(}\PYG{l+m+mf}{9.2}\PYG{p}{,}\PYG{o}{.}\PYG{l+m+mi}{8}\PYG{p}{)}
\PYG{n}{sigma} \PYG{o}{=} \PYG{n}{b}\PYG{o}{.}\PYG{n}{distributions}\PYG{o}{.}\PYG{n}{gamma}\PYG{p}{(}\PYG{l+m+mf}{1.}\PYG{p}{,}\PYG{l+m+mf}{1.}\PYG{o}{/}\PYG{l+m+mf}{4.8}\PYG{p}{)}
\PYG{n}{k1} \PYG{o}{=} \PYG{n}{b}\PYG{o}{.}\PYG{n}{distributions}\PYG{o}{.}\PYG{n}{gamma}\PYG{p}{(}\PYG{l+m+mf}{1.}\PYG{p}{,}\PYG{l+m+mf}{1.}\PYG{o}{/}\PYG{l+m+mi}{3}\PYG{p}{)}
\PYG{n}{k2} \PYG{o}{=} \PYG{n}{b}\PYG{o}{.}\PYG{n}{distributions}\PYG{o}{.}\PYG{n}{gamma}\PYG{p}{(}\PYG{l+m+mf}{1.}\PYG{p}{,}\PYG{l+m+mf}{1.}\PYG{o}{/}\PYG{l+m+mf}{8.}\PYG{p}{)}

\PYG{c}{\PYGZsh{} Collect the distributions}
\PYG{n}{priors} \PYG{o}{=} \PYG{n}{b}\PYG{o}{.}\PYG{n}{distributions}\PYG{o}{.}\PYG{n}{parameter\PYGZus{}collection}\PYG{p}{(}\PYG{n}{e1}\PYG{p}{,}\PYG{n}{e2}\PYG{p}{,}\PYG{n}{sigma}\PYG{p}{,}\PYG{n}{k1}\PYG{p}{,}\PYG{n}{k2}\PYG{p}{)}

\PYG{c}{\PYGZsh{} Loop over all the traces}
\PYG{k}{for} \PYG{n}{i} \PYG{o+ow}{in} \PYG{n+nb}{range}\PYG{p}{(}\PYG{n}{data}\PYG{o}{.}\PYG{n}{attr}\PYG{o}{.}\PYG{n}{n\PYGZus{}traces}\PYG{p}{)}\PYG{p}{:}

        \PYG{c}{\PYGZsh{} Log the priors for this molecule in the SMD}
        \PYG{n}{data} \PYG{o}{=} \PYG{n}{b}\PYG{o}{.}\PYG{n}{smd}\PYG{o}{.}\PYG{n}{add}\PYG{o}{.}\PYG{n}{priors}\PYG{p}{(}\PYG{n}{data}\PYG{p}{,}\PYG{n}{i}\PYG{p}{,}\PYG{n}{priors}\PYG{p}{)}

        \PYG{c}{\PYGZsh{} Setup the MCMC sampler with 50 walkers and 8 CPUs on the FRET data}
        \PYG{n}{nwalkers} \PYG{o}{=} \PYG{l+m+mi}{50}
        \PYG{n}{sampler}\PYG{p}{,} \PYG{n}{initial\PYGZus{}positions} \PYG{o}{=} \PYG{n}{b}\PYG{o}{.}\PYG{n}{mcmc}\PYG{o}{.}\PYG{n}{setup}\PYG{p}{(}\PYG{n}{data}\PYG{o}{.}\PYG{n}{data}\PYG{p}{[}\PYG{n}{i}\PYG{p}{]}\PYG{o}{.}\PYG{n}{values}\PYG{o}{.}\PYG{n}{FRET}\PYG{p}{,}
                \PYG{n}{b}\PYG{o}{.}\PYG{n}{smd}\PYG{o}{.}\PYG{n}{read}\PYG{o}{.}\PYG{n}{priors}\PYG{p}{(}\PYG{n}{data}\PYG{p}{,}\PYG{n}{i}\PYG{p}{)}\PYG{p}{,} \PYG{n}{tau}\PYG{p}{,}
                \PYG{n}{nwalkers}\PYG{p}{,} \PYG{n}{initialize}\PYG{o}{=}\PYG{l+s}{\PYGZsq{}}\PYG{l+s}{rvs}\PYG{l+s}{\PYGZsq{}}\PYG{p}{,} \PYG{n}{threads}\PYG{o}{=}\PYG{l+m+mi}{8}\PYG{p}{)}

        \PYG{c}{\PYGZsh{} Run the MCMC: burn\PYGZhy{}in first, then production}
        \PYG{n}{sampler}\PYG{p}{,} \PYG{n}{burned\PYGZus{}positions} \PYG{o}{=} \PYG{n}{b}\PYG{o}{.}\PYG{n}{mcmc}\PYG{o}{.}\PYG{n}{burn\PYGZus{}in}\PYG{p}{(}\PYG{n}{sampler}\PYG{p}{,}
                \PYG{n}{initial\PYGZus{}positions}\PYG{p}{,} \PYG{n}{nsteps}\PYG{o}{=}\PYG{l+m+mi}{100}\PYG{p}{)}
        \PYG{n}{sampler} \PYG{o}{=} \PYG{n}{b}\PYG{o}{.}\PYG{n}{mcmc}\PYG{o}{.}\PYG{n}{run}\PYG{p}{(}\PYG{n}{sampler}\PYG{p}{,}\PYG{n}{burned\PYGZus{}positions}\PYG{p}{,}\PYG{n}{nsteps}\PYG{o}{=}\PYG{l+m+mi}{1000}\PYG{p}{,}\PYG{n}{timer}\PYG{o}{=}\PYG{n+nb+bp}{True}\PYG{p}{)}

        \PYG{c}{\PYGZsh{} Calculate acceptance ratio and autocorrelation times}
        \PYG{n}{largest\PYGZus{}autocorrelation\PYGZus{}time} \PYG{o}{=} \PYG{n}{b}\PYG{o}{.}\PYG{n}{mcmc}\PYG{o}{.}\PYG{n}{chain\PYGZus{}statistics}\PYG{p}{(}\PYG{n}{sampler}\PYG{p}{)}

        \PYG{c}{\PYGZsh{} Save this data}
        \PYG{n}{data} \PYG{o}{=} \PYG{n}{b}\PYG{o}{.}\PYG{n}{smd}\PYG{o}{.}\PYG{n}{add}\PYG{o}{.}\PYG{n}{mcmc}\PYG{p}{(}\PYG{n}{data}\PYG{p}{,}\PYG{n}{i}\PYG{p}{,}\PYG{n}{sampler}\PYG{p}{)}
        \PYG{n}{b}\PYG{o}{.}\PYG{n}{smd}\PYG{o}{.}\PYG{n}{save}\PYG{p}{(}\PYG{l+s}{\PYGZsq{}}\PYG{l+s}{data.smd}\PYG{l+s}{\PYGZsq{}}\PYG{p}{,}\PYG{n}{data}\PYG{p}{)}
\end{Verbatim}


\subsection{Plot BIASD-MCMC Results Using Corner}
\label{getstarted:plot-biasd-mcmc-results-using-corner}
Use corner to plot the 5-D space of the posterior sampled by MCMC. Read about corner \href{http://corner.readthedocs.io/en/latest/}{here}.

\begin{Verbatim}[commandchars=\\\{\}]
\PYG{k+kn}{import} \PYG{n+nn}{numpy} \PYG{k+kn}{as} \PYG{n+nn}{np}
\PYG{k+kn}{import} \PYG{n+nn}{biasd} \PYG{k+kn}{as} \PYG{n+nn}{b}

\PYG{c}{\PYGZsh{} Load some SMD format data}
\PYG{n}{data} \PYG{o}{=} \PYG{n}{b}\PYG{o}{.}\PYG{n}{smd}\PYG{o}{.}\PYG{n}{load}\PYG{p}{(}\PYG{l+s}{\PYGZsq{}}\PYG{l+s}{data.smd}\PYG{l+s}{\PYGZsq{}}\PYG{p}{)}

\PYG{c}{\PYGZsh{} Get read the sampler results for the first trace (0)}
\PYG{n}{sampler\PYGZus{}results} \PYG{o}{=} \PYG{n}{b}\PYG{o}{.}\PYG{n}{smd}\PYG{o}{.}\PYG{n}{read}\PYG{o}{.}\PYG{n}{mcmc}\PYG{p}{(}\PYG{n}{s}\PYG{p}{,}\PYG{l+m+mi}{0}\PYG{p}{)}

\PYG{c}{\PYGZsh{} Get the correlated samples}
\PYG{n}{samples\PYGZus{}corr} \PYG{o}{=} \PYG{n}{sampler\PYGZus{}results}\PYG{o}{.}\PYG{n}{chain}

\PYG{c}{\PYGZsh{} Remove some really bad samples}
\PYG{n}{cut} \PYG{o}{=} \PYG{n}{sampler\PYGZus{}results}\PYG{o}{.}\PYG{n}{lnprobability} \PYG{o}{\PYGZlt{}} \PYG{l+m+mf}{0.}
\PYG{n}{samples\PYGZus{}corr} \PYG{o}{=} \PYG{n}{samples\PYGZus{}corr}\PYG{p}{[}\PYG{o}{\PYGZti{}}\PYG{n}{cut}\PYG{p}{]}\PYG{o}{.}\PYG{n}{reshape}\PYG{p}{(}\PYG{p}{(}\PYG{o}{\PYGZhy{}}\PYG{l+m+mi}{1}\PYG{p}{,}\PYG{l+m+mi}{5}\PYG{p}{)}\PYG{p}{)}

\PYG{c}{\PYGZsh{} Get the uncorrelated samples (previously calculated)}
\PYG{n}{samples\PYGZus{}uncorr} \PYG{o}{=} \PYG{n}{sampler\PYGZus{}results}\PYG{o}{.}\PYG{n}{samples}

\PYG{c}{\PYGZsh{} Plot corner plots}
\PYG{n}{f} \PYG{o}{=} \PYG{n}{b}\PYG{o}{.}\PYG{n}{mcmc}\PYG{o}{.}\PYG{n}{plot\PYGZus{}corner}\PYG{p}{(}\PYG{n}{samples\PYGZus{}corr}\PYG{p}{)}
\PYG{n}{f} \PYG{o}{=} \PYG{n}{b}\PYG{o}{.}\PYG{n}{mcmc}\PYG{o}{.}\PYG{n}{plot\PYGZus{}corner}\PYG{p}{(}\PYG{n}{samples\PYGZus{}uncorr}\PYG{p}{)}
\end{Verbatim}


\subsection{Plot BIASD-MCMC Results Using Viewer}
\label{getstarted:plot-biasd-mcmc-results-using-viewer}
From the above example with corner, you can use the built in distribution viewer to explore the marginalized BIASD posterior distribution.

\begin{Verbatim}[commandchars=\\\{\}]
\PYG{c}{\PYGZsh{} Create a collection of distributions from the marginalized samples}
\PYG{n}{posterior} \PYG{o}{=} \PYG{n}{b}\PYG{o}{.}\PYG{n}{mcmc}\PYG{o}{.}\PYG{n}{create\PYGZus{}posterior\PYGZus{}collection}\PYG{p}{(}\PYG{n}{samples\PYGZus{}uncorr}\PYG{p}{,}\PYG{n}{priors}\PYG{p}{)}

\PYG{c}{\PYGZsh{} View the marginalized posterior in a biasd.distribution.viewer}
\PYG{n}{b}\PYG{o}{.}\PYG{n}{distributions}\PYG{o}{.}\PYG{n}{viewer}\PYG{p}{(}\PYG{n}{posterior}\PYG{p}{)}
\end{Verbatim}


\subsection{BIASD + Laplace Approximation}
\label{getstarted:biasd-laplace-approximation}
You can also use the Laplace approximation to approximate the posterior distribution as a multidimensional gaussian centered the the maximum a postiori (MAP) value of the distribution. The advantage is that it is probably faster than MCMC, however, it is definitely an approximation.

\begin{Verbatim}[commandchars=\\\{\}]
\PYG{k+kn}{import} \PYG{n+nn}{numpy} \PYG{k+kn}{as} \PYG{n+nn}{np}
\PYG{k+kn}{import} \PYG{n+nn}{biasd} \PYG{k+kn}{as} \PYG{n+nn}{b}

\PYG{c}{\PYGZsh{} Load some SMD format data}
\PYG{n}{data} \PYG{o}{=} \PYG{n}{b}\PYG{o}{.}\PYG{n}{smd}\PYG{o}{.}\PYG{n}{load}\PYG{p}{(}\PYG{l+s}{\PYGZsq{}}\PYG{l+s}{data.smd}\PYG{l+s}{\PYGZsq{}}\PYG{p}{)}
\PYG{n}{tau} \PYG{o}{=} \PYG{l+m+mf}{0.1}

\PYG{c}{\PYGZsh{} Setup the prior distributions}
\PYG{n}{e1} \PYG{o}{=} \PYG{n}{b}\PYG{o}{.}\PYG{n}{distributions}\PYG{o}{.}\PYG{n}{beta}\PYG{p}{(}\PYG{l+m+mf}{1.}\PYG{p}{,}\PYG{l+m+mf}{9.}\PYG{p}{)}
\PYG{n}{e2} \PYG{o}{=} \PYG{n}{b}\PYG{o}{.}\PYG{n}{distributions}\PYG{o}{.}\PYG{n}{beta}\PYG{p}{(}\PYG{l+m+mf}{9.}\PYG{p}{,}\PYG{l+m+mf}{1.}\PYG{p}{)}
\PYG{n}{sigma} \PYG{o}{=} \PYG{n}{b}\PYG{o}{.}\PYG{n}{distributions}\PYG{o}{.}\PYG{n}{gamma}\PYG{p}{(}\PYG{l+m+mf}{2.}\PYG{p}{,}\PYG{l+m+mf}{2.}\PYG{o}{/}\PYG{o}{.}\PYG{l+m+mo}{05}\PYG{p}{)}
\PYG{n}{k1} \PYG{o}{=} \PYG{n}{b}\PYG{o}{.}\PYG{n}{distributions}\PYG{o}{.}\PYG{n}{gamma}\PYG{p}{(}\PYG{l+m+mf}{20.}\PYG{p}{,}\PYG{l+m+mf}{20.}\PYG{o}{/}\PYG{l+m+mf}{3.}\PYG{p}{)}
\PYG{n}{k2} \PYG{o}{=} \PYG{n}{b}\PYG{o}{.}\PYG{n}{distributions}\PYG{o}{.}\PYG{n}{gamma}\PYG{p}{(}\PYG{l+m+mf}{20.}\PYG{p}{,}\PYG{l+m+mf}{20.}\PYG{o}{/}\PYG{l+m+mf}{8.}\PYG{p}{)}
\PYG{n}{priors} \PYG{o}{=} \PYG{n}{b}\PYG{o}{.}\PYG{n}{distributions}\PYG{o}{.}\PYG{n}{parameter\PYGZus{}collection}\PYG{p}{(}\PYG{n}{e1}\PYG{p}{,}\PYG{n}{e2}\PYG{p}{,}\PYG{n}{sigma}\PYG{p}{,}\PYG{n}{k1}\PYG{p}{,}\PYG{n}{k2}\PYG{p}{)}

\PYG{c}{\PYGZsh{} Loop over the traces}
\PYG{k}{for} \PYG{n}{i} \PYG{o+ow}{in} \PYG{n+nb}{range}\PYG{p}{(}\PYG{n}{data}\PYG{o}{.}\PYG{n}{attr}\PYG{o}{.}\PYG{n}{n\PYGZus{}traces}\PYG{p}{)}\PYG{p}{:}

        \PYG{c}{\PYGZsh{} Perform Laplace approximation}
        \PYG{n}{d} \PYG{o}{=} \PYG{n}{data}\PYG{o}{.}\PYG{n}{data}\PYG{p}{[}\PYG{n}{i}\PYG{p}{]}\PYG{o}{.}\PYG{n}{values}\PYG{o}{.}\PYG{n}{FRET}
        \PYG{n}{lp} \PYG{o}{=} \PYG{n}{b}\PYG{o}{.}\PYG{n}{laplace}\PYG{o}{.}\PYG{n}{laplace\PYGZus{}approximation}\PYG{p}{(}\PYG{n}{d}\PYG{p}{,}\PYG{n}{priors}\PYG{p}{,}\PYG{n}{tau}\PYG{p}{)}

        \PYG{c}{\PYGZsh{} Add the priors and results to the SMD}
        \PYG{n}{data} \PYG{o}{=} \PYG{n}{b}\PYG{o}{.}\PYG{n}{utils}\PYG{o}{.}\PYG{n}{smd}\PYG{o}{.}\PYG{n}{add\PYGZus{}priors}\PYG{p}{(}\PYG{n}{data}\PYG{p}{,}\PYG{n}{i}\PYG{p}{,}\PYG{n}{priors}\PYG{p}{)}
        \PYG{n}{data} \PYG{o}{=} \PYG{n}{b}\PYG{o}{.}\PYG{n}{utils}\PYG{o}{.}\PYG{n}{smd}\PYG{o}{.}\PYG{n}{add\PYGZus{}laplace\PYGZus{}posterior}\PYG{p}{(}\PYG{n}{data}\PYG{p}{,}\PYG{n}{i}\PYG{p}{,}\PYG{n}{lp}\PYG{p}{)}

        \PYG{c}{\PYGZsh{} Calculate the moment\PYGZhy{}matched, marginalized posterior}
        \PYG{c}{\PYGZsh{} of the same form as the prior distributions}
        \PYG{n}{lp}\PYG{o}{.}\PYG{n}{transform}\PYG{p}{(}\PYG{n}{priors}\PYG{p}{)}
        \PYG{n}{data} \PYG{o}{=} \PYG{n}{b}\PYG{o}{.}\PYG{n}{utils}\PYG{o}{.}\PYG{n}{smd}\PYG{o}{.}\PYG{n}{add\PYGZus{}posterior}\PYG{p}{(}\PYG{n}{data}\PYG{p}{,}\PYG{n}{i}\PYG{p}{,}\PYG{n}{lp}\PYG{o}{.}\PYG{n}{posterior}\PYG{p}{)}

\PYG{c}{\PYGZsh{} Save the results}
\PYG{n}{b}\PYG{o}{.}\PYG{n}{utils}\PYG{o}{.}\PYG{n}{smd}\PYG{o}{.}\PYG{n}{save}\PYG{p}{(}\PYG{l+s}{\PYGZsq{}}\PYG{l+s}{data.smd}\PYG{l+s}{\PYGZsq{}}\PYG{p}{,}\PYG{n}{data}\PYG{p}{)}
\end{Verbatim}


\subsection{My Baseline and BIASD...}
\label{getstarted:my-baseline-and-biasd}
If your baseline is crazy, BIASD will not work very well. In the \code{./utils} directory there a method to try to integrate out a baseline that follows a Brownian diffusion process. Since this implementation is built upon a Gaussian mixture model, it's probably inappropriate to use this when there is a lot of blurring.

\begin{Verbatim}[commandchars=\\\{\}]
\PYG{c}{\PYGZsh{} Load data}
\PYG{n}{data} \PYG{o}{=} \PYG{n}{b}\PYG{o}{.}\PYG{n}{smd}\PYG{o}{.}\PYG{n}{load}\PYG{p}{(}\PYG{l+s}{\PYGZsq{}}\PYG{l+s}{data.smd}\PYG{l+s}{\PYGZsq{}}\PYG{p}{)}

\PYG{c}{\PYGZsh{} Let\PYGZsq{}s remove the baseline of the first trace}
\PYG{n}{d} \PYG{o}{=} \PYG{n}{data}\PYG{o}{.}\PYG{n}{data}\PYG{p}{[}\PYG{l+m+mi}{0}\PYG{p}{]}\PYG{o}{.}\PYG{n}{values}\PYG{o}{.}\PYG{n}{FRET}
\PYG{n}{baseline\PYGZus{}result} \PYG{o}{=} \PYG{n}{b}\PYG{o}{.}\PYG{n}{utils}\PYG{o}{.}\PYG{n}{baseline}\PYG{o}{.}\PYG{n}{remove\PYGZus{}baseline}\PYG{p}{(}\PYG{n}{d}\PYG{p}{)}
\PYG{n}{data} \PYG{o}{=} \PYG{n}{b}\PYG{o}{.}\PYG{n}{utils}\PYG{o}{.}\PYG{n}{smd}\PYG{o}{.}\PYG{n}{add\PYGZus{}baseline}\PYG{p}{(}\PYG{n}{data}\PYG{p}{,}\PYG{l+m+mi}{0}\PYG{p}{,}\PYG{n}{baseline\PYGZus{}result}\PYG{p}{)}

\PYG{c}{\PYGZsh{} Save the results}
\PYG{n}{b}\PYG{o}{.}\PYG{n}{utils}\PYG{o}{.}\PYG{n}{smd}\PYG{o}{.}\PYG{n}{save}\PYG{p}{(}\PYG{l+s}{\PYGZsq{}}\PYG{l+s}{data.smd}\PYG{l+s}{\PYGZsq{}}\PYG{p}{,}\PYG{n}{data}\PYG{p}{)}

\PYG{c}{\PYGZsh{} Subtract off the baseline for use in some other calculation}
\PYG{n}{d} \PYG{o}{\PYGZhy{}}\PYG{o}{=} \PYG{n}{baseline\PYGZus{}result}\PYG{o}{.}\PYG{n}{baseline}
\end{Verbatim}


\section{Compiling the Likelihood}
\label{compileguide:compiling-the-likelihood}\label{compileguide::doc}\label{compileguide:compileguide}

\subsection{Background}
\label{compileguide:background}
The BIASD log-likelihood function is something like
\begin{align*}\begin{aligned}
\begin{split}ln(\mathcal{L}) \sim \sum\limits_t ln \left( \delta(f) + \delta(1-f) + \int\limits_0^1 df \cdot \rm{blurring} \right)\end{split}\end{aligned}\end{align*}
Unfortunately, the integral in the logarithm makes it difficult to compute. It is the rate limiting step for this calculation, which is quite slow in Python. Therefore, this package comes with the log-likelihood function written in  C, and also in CUDA. There are three versions in the \code{./src} directory. One is in pure C -- it should be fairly straight forward to compile. The second is written in C with the \href{https://www.gnu.org/software/gsl/}{GNU Science Library (GSL)} -- it's slightly faster, but requires having installed GSL. The third is in CUDA, which allows the calculations to be performed on NVIDIA GPUs. You can use any of the above if compiled, or a version written in Python if you don't want to compile anything.


\subsection{How to Compile}
\label{compileguide:how-to-compile}
There's a Makefile included in the package that will allow you to easily compile all of the libraries necessary to calculate BIASD likelihoods. First, to download GSL, go to their \href{ftp://ftp.gnu.org/gnu/gsl/}{FTP site} and download the latest version. Un-pack it, then in the terminal, navigate to the directory using \code{cd} and type

\begin{Verbatim}[commandchars=\\\{\}]
./configure
make
make install
\end{Verbatim}

Now, even if you didn't install GSL, you can compile the BIASD likelihood functions. In the terminal, move to the BIASD directory using \code{cd}, and make them with

\begin{Verbatim}[commandchars=\\\{\}]
make
\end{Verbatim}

Some might fail, for instance if you don't have a CUDA-enabled GPU, but you'll compile as many as possible into the \code{./lib} directory.


\subsection{Testing Speed}
\label{compileguide:testing-speed}
To get a feeling for how long it takes the various versions of the BIASD likelihood function to execute, you can use the test function in the likelihood module. For instance, try

\begin{Verbatim}[commandchars=\\\{\}]
\PYG{k+kn}{import} \PYG{n+nn}{biasd}

\PYG{c}{\PYGZsh{} Switch to the Python version}
\PYG{n}{biasd}\PYG{o}{.}\PYG{n}{likelihood}\PYG{o}{.}\PYG{n}{use\PYGZus{}python\PYGZus{}ll}\PYG{p}{(}\PYG{p}{)}

\PYG{c}{\PYGZsh{} Run the test 10 times, for 5000 datapoints}
\PYG{n}{biasd}\PYG{o}{.}\PYG{n}{likelihood}\PYG{o}{.}\PYG{n}{test\PYGZus{}speed}\PYG{p}{(}\PYG{l+m+mi}{10}\PYG{p}{,}\PYG{l+m+mi}{5000}\PYG{p}{)}

\PYG{c}{\PYGZsh{} Switch to the C version and test}
\PYG{c}{\PYGZsh{} Note: will default to GSL over pure C}
\PYG{n}{biasd}\PYG{o}{.}\PYG{n}{likelihood}\PYG{o}{.}\PYG{n}{use\PYGZus{}C\PYGZus{}ll}\PYG{p}{(}\PYG{p}{)}
\PYG{n}{biasd}\PYG{o}{.}\PYG{n}{likelihood}\PYG{o}{.}\PYG{n}{test\PYGZus{}speed}\PYG{p}{(}\PYG{l+m+mi}{10}\PYG{p}{,}\PYG{l+m+mi}{5000}\PYG{p}{)}

\PYG{c}{\PYGZsh{} Switch to the CUDA version and test}
\PYG{n}{biasd}\PYG{o}{.}\PYG{n}{likelihood}\PYG{o}{.}\PYG{n}{use\PYGZus{}CUDA\PYGZus{}ll}\PYG{p}{(}\PYG{p}{)}
\PYG{n}{biasd}\PYG{o}{.}\PYG{n}{likelihood}\PYG{o}{.}\PYG{n}{test\PYGZus{}speed}\PYG{p}{(}\PYG{l+m+mi}{10}\PYG{p}{,}\PYG{l+m+mi}{5000}\PYG{p}{)}
\end{Verbatim}

The actual execution time depends upon the rate constants, but Python is \textasciitilde{} 1 ms, C with GSL is around \textasciitilde{}40 us, and CUDA (when you have many datapoints) is \textasciitilde{} 5 us.



\renewcommand{\indexname}{Index}
\printindex
\end{document}
